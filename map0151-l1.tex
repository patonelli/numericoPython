\documentclass[12pt]{article}
\usepackage{amsfonts, amsmath, amssymb}
\usepackage[brazil]{babel}
\usepackage{graphicx}
\usepackage[T1]{fontenc}
\usepackage{lmodern}

\parindent=0pt

 \addtolength{\textheight}{3.5cm}
 \addtolength{\oddsidemargin}{-1cm}
 \addtolength{\evensidemargin}{-1cm}
 \addtolength{\textwidth}{2cm}
 \addtolength{\topmargin}{-2.0cm}
\newcounter{questao}
\newcommand{\quest}{\stepcounter{questao}{\bf \arabic{questao}.\ }}

\begin{document}
\hrule
 {  \sf  Lista de Exerc�cios- MAP151-1  \hfill \fbox{L1-2016}}
\hrule

\vspace{0.5cm}

\thispagestyle{empty}
\fontsize{14}{16}\selectfont

\quest Escreva cada um dos n�meros abaixo, expressos na base decimal, na base
bin�ria, octal e hexadecimal e na forma normalizada em ponto flutuante, com
mantissa de tr�s d�gitos.
\begin{itemize}
\item $3.275$
\item $2/3$
\end{itemize}

\vspace{0.3cm}

\quest  Considere $\beta =8$ a base do sistema de n�meros e o conjunto de 
n�meros de m�quina $\mathbb{M} = \{0.d_1\cdots d_8 \times 8^{e}\}$ onde o
m�ximo do valor absoluto de $e$ � $[777]_8$. Quantos n�meros tem este conjunto de n�meros de m�quina. Qual � o arredondamento de $\pi$ neste conjunto?

\vspace{0.3cm}

\quest Resolva o sistema linear com o m�todo da elimina��o de Gauss, usando aritm�tica de ponto flutuante com mantissa de dois d�gitos. 
\begin{gather*} 0.01 x + 1.7 y + 2.1z  = 15 \\
   9y - 2.3 z = 2.1 \\
  3z = 1\end{gather*}
Compare com a solu��o real.

\vspace{0.3cm}

\quest .
Use o m�todo da elimina��o de Gauss para encontrar a inversa da matriz $A$ dada abaixo:
\begin{gather*}
  A=
  \begin{pmatrix}
    2.3 & 0 & -1 & 5.5 \\
    0 & 4 & -3 & 2.1 \\
   -1 & 5 & 6.5 & 0 \\
    2 & 0 & 0 & -1
  \end{pmatrix}
\end{gather*}

\vspace{0.3cm}

\quest  Escrever a representa��o na base $2$ dos seguintes n�meros que
est�o na base $10$
\begin{gather*}
  \text{a) } 0.125 \hspace{1cm} 
\text{ b) } 0.1  \hspace{1cm} 
\text{ c) } 0.05   \hspace{1cm} 
\text{ d) } 5.6
\end{gather*}

\vspace{0.3cm}

\quest Encontrar a equa��o da par�bola $y= ax^2 + bx + c$ que passa pelos pontos
$(-1,4)$, $(1,8)$, $(-2,23)$.

\vspace{0.3cm}

\quest  Resolver os seguinte sistema usando o m�todo de elimina��o de Gauss com pivota��o.
\begin{gather*}
  4x_1 - x_2 -x_3  = 5\\
-x_1 +4x_2 -x_4  =  -3 \\
-x_1 +4x_3 -x_4  = -7 \\
-x_2 -x_3 +4x_4  =  9 \\
\end{gather*}

\vspace{0.3cm}

\quest Encontrar a decomposi��o $LU$ da matriz abaixo:
\begin{gather*}
  A=
  \begin{pmatrix}
    2 & 1 & 3 & 0 \\
-2 &-2&-4&5 \\
5 & 0.5 & 7.5 & 11 \\
2 & -1 & 3 & 13
  \end{pmatrix}
\end{gather*}


\vspace{0.3cm}

\quest  Usando o m�todo da elimina��o de Gauss, encontre a matriz C que satisfaz a equa��o:
\begin{gather*}
  \begin{pmatrix}
    2 & 0 & -1 \\
    1 & 1 & 5 \\
   -1 & -2 & 10 
  \end{pmatrix}C =
  \begin{pmatrix}
    1 & 0 \\
6 & 21 \\
9 & 40
  \end{pmatrix}
\end{gather*}
 

\vspace{0.3cm}

\quest  Achar a decomposi��o $LU$ da matriz
\begin{gather*}
  A=
  \begin{pmatrix}
    7 & 0 & -3 \\
   -14 & 2 & 8 \\
    21 & -2 & 16
  \end{pmatrix}
\end{gather*}


\end{document}

%%% Local Variables: 
%%% mode: latex
%%% TeX-master: t
%%% End: 

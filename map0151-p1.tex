\documentclass[12pt]{article}
\usepackage{amsfonts, amsmath, amssymb}
\usepackage[brazil]{babel}
\usepackage{graphicx}
\usepackage[T1]{fontenc}
\usepackage{lmodern}

\parindent=0pt

 \addtolength{\textheight}{3.5cm}
 \addtolength{\oddsidemargin}{-1cm}
 \addtolength{\evensidemargin}{-1cm}
 \addtolength{\textwidth}{2cm}
 \addtolength{\topmargin}{-2.0cm}
\newcounter{questao}
\newcommand{\quest}{\stepcounter{questao}{\bf \arabic{questao}.\ }}

\begin{document}
\hrule
 {  \sf  Prova - MAP151-1  \hfill \fbox{P1-2016}}
\hrule

\vspace{0.5cm}

\thispagestyle{empty}
\fontsize{14}{16}\selectfont

\quest Vamos fixar a base $B=4.$ No primeiro item temos um n�mero escrito na base $B,$ escreva-o na base 10. No segundo item o inverso, temos o n�mero na base $10$, passe-o para a base $B$.
\begin{itemize}
\item $1001.1$
\item $75.5$
\end{itemize}

\vspace{0.3cm}

\quest Definimos $a=32.56\times 10^{-1}$ e $b=3.85.$ Escreva estes n�meros na forma normal em ponto flutuante. Calcule $a*b$ com dois algarismos significativos e d� o resultado tamb�m na forma normal.

\vspace{0.3cm}

\quest Achar a decomposi��o \textbf{LU} da matriz
$$ A=
\begin{pmatrix}
  3 & -2 & -1 \\ 0 & 5 & 1 \\ 3 & 8 & 4
\end{pmatrix}$$

\vspace{0.3cm}

\quest Resolver o sistema linear abaixo usando o MEG com pivota��o e escrevendo a matriz resultante ap�s cada passo, use a aritm�tica com dois algarismos significativos.
\begin{eqnarray}
  \label{eq:1}
  0.2x + y +z &=&1.2 \\
x + 0.33y + z &=& 0 \\
x + 0.25z &=&0.5
\end{eqnarray}


\vspace{0.3cm}

\quest As matrizes $A$ e $A^{(1)}$ abaixo s�o as matrizes de coeficientes de sistemas lineares. $A^{(1)}$ � o resultado da aplica��o do primeiro passo do m�todo de elimina��o de Gauss, com pivota��o, na matriz $A$. Quais s�o as matrizes elementares envolvidas no processo e ache a matriz $P$ tal que $A^{(1)}=PA$.
\begin{gather*}
  A=
  \begin{pmatrix}
    1&2&0 \\ 2&1&1 \\1&-1&-1
  \end{pmatrix}\to A^{(1)}=
  \begin{pmatrix}
    2&1&1\\ 0 & 1.5 & -0.5 \\ 0&-1.5&-1.5
  \end{pmatrix}
\end{gather*}


\end{document}

%%% Local Variables: 
%%% mode: latex
%%% TeX-master: t
%%% End: 

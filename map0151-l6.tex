\documentclass[12pt]{article}
\usepackage{amsfonts, amsmath, amssymb}
\usepackage[brazil]{babel}
\usepackage{graphicx}
\usepackage[T1]{fontenc}
\usepackage{lmodern}

\parindent=0pt

 \addtolength{\textheight}{3.5cm}
 \addtolength{\oddsidemargin}{-1cm}
 \addtolength{\evensidemargin}{-1cm}
 \addtolength{\textwidth}{2cm}
 \addtolength{\topmargin}{-2.0cm}
\newcounter{questao}
\newcommand{\quest}{\stepcounter{questao}{\bf \arabic{questao}.\ }}

\begin{document}
\hrule
 {  \sf  Lista de Exerc�cios- MAP151-6  \hfill \fbox{L6-2016}}
\hrule

\vspace{0.5cm}

\thispagestyle{empty}
\fontsize{14}{16}\selectfont

\quest Seja $\mathbf{T}=\{ (x_0,y_0), \dots , (x_k,y_k)\}$ uma tabela regular. Mostre que os polin�mios de Lagrange desta tabela s�o linearmente independentes, isto �, $\sum_{i=0}^k a_iL_i(x) = 0$ se, e somente se, $a_i=0$ para todo �ndice $i$. Se $p(x)$ � um polin�mio de grau menor ou igual a $k$, ent�o ele se escreve de uma �nica forma como combina��o linear dos polin�mios de Lagrange. Como s�o as coordenadas?

\vspace{0.3cm}

\quest Ache o polin�mio interpolador na forma de Lagrange da tabela

$\{(0,1),(1,3), (5,2), (3,1)\}$

\vspace{0.3cm}

\quest Na tabela $\{(x_0,y_0),(x_1,y_1),(x_2,y_2)\}$, os elementos $x_i$ s�o as raizes do polin�mio $q(x)=2x^3-3x^2 -2x +3$ e o polin�mio interpolador da tabela � $p(x) = x^2 + 2x$. Adicionando-se o ponto $(0,1)$ � tabela original, qual � o novo polin�mio interpolador?

\vspace{0.3cm}

\quest Fa�a a tabela de diferen�as divididas e escreva o polin�mio interpolador na forma de Newton da tabela abaixo
$\{(0,2),(1,0), (5,2), (3,1)\}$


\vspace{0.3cm}

\quest  Considere a tabela da fun��o $f(x) = \frac{\exp{(-x^2/2)}}{\sqrt{2\pi}}$

    \begin{tabular}{||c|c|c|c|c||}
     $ -1$&$-0.5$&$0$&$0.5$&$1$ \\ \hline
      $0.242$&$0.352$&$0.399$&$0.352$&$0.242$
    \end{tabular}
    
Fa�a uma estimativa de $f(0.3)$ e ache uma estimativa do erro cometido.

\vspace{0.3cm}

\quest Calcule a integral
$$ \int_{-1}^1 \frac{x+1}{x^2 + 2}dx $$
usando
\begin{itemize}
\item o m�todo do trap�zio com duas repeti��es
\item o m�todo de Simpson simples
\item o m�todo de Simpson com duas repeti��es
\end{itemize}

\vspace{0.3cm}

\quest Ache os quatro primeiros polin�mios ortogonais em rela��o ao produto interno $$ <f,g> = \int_{-1}^1f(x)g(x)dx. $$ Escreva o polin�mio $p(x)=x^2$ como combina��o linear dos polin�mios desta fam�lia.


\end{document}

%%% Local Variables: 
%%% mode: latex
%%% TeX-master: t
%%% End: 
